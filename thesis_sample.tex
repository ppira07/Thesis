\documentclass[11pt,a4j]{jsarticle}

%本ファイル使用パッケージ
\usepackage{graphicx}

%卒論スタイルファイル
\usepackage{sty/ib_thesis}

%TeX数式用
%\usepackage{amsmath}
\usepackage{ascmac}
%\usepackage{graphix}

%argmax定義
\newcommand{\argmax}{\mathop{\rm arg~max}\limits}

%コメントアウト
\usepackage{comment}

%表紙パラメータ設定
%
\author{平野 万由子}     %本人名
\supervisor{萩原 信吾} %指導教員名
\fyear{27}             %所属年度
\submit{28}{1}{31}    %提出和暦,月,日
\title{ナイーブベイズ分類を用いた文の感情推定と顔文字自動付与システムの提案} %論文タイトル


%ドキュメントスタート
\begin{document}
 \pagestyle{empty}
 \clearpage

 %表紙の生成
 \maketitle

 %アブストラクトはもし中国文字・朝鮮文字が必要なら,WORDで書く(中・朝文字をだすのはめんどい)
 %英語でよければ次のように

 %日本語アブストラクト
 \jabstract{
近年、インターネットの普及により、多くのユーザが SNS (Social Networking Service)を利用している。
SNS では、テキストベースのやりとりが常套化しており、さらに簡潔な短文でやりとりをするチャット形式のコミュニケーションツールを利用するユーザが増加している。
しかし、テキストベースのやりとりでは、音声や表情といった非言語情報が欠落しがちである。
そのため、テキストの書き手と読み手の間に認識の齟齬が発生する。
そこで本研究では、インターネット上のテキストの感情を推定し、テキストの感情を補助するための顔文字を付与するシステムを提案する。
感情の推定では、感情語辞書を用いて品詞単位に感情を推定し、ナイーブベイズ分類器を用いて感情のスコアを算出する。
算出した感情のスコアを元に、顔文字辞書を用いて、推定した文の感情に見合った顔文字を付与する。
本研究では、顔文字を`喜'、`悲'、`驚'、`不安'、`期待'、`怒'、`平静'の7感情に分類した。}

 %英語アブストラクト
 \eabstract{Many users are resently using SNS (Social Networking Service).
On SNS, text-based exchange has been normaly used. 
Furthermore, the number of users who are using communication tools for chatting has been increasing.
In the case they write short sentences, the non-language information as sounds and expressions are lacked.
Therefore, the messages make the gap of between writers and readers.
In this study, we propose the system which estimates the emotion of a sentence, and give an emoticon to recivers on the trols.
To estimate emotion, we divide a sentence into a part of speech with emotional words dictionary, then we calculate feeling score with a naivebayes classifier.
Based on the result of calculated score, we gives an emoticon which is suitable to the sentence with emoticon dictionary.
In adittion, we divide the emoticons into 7 feelings as `happiness', `sadness', `surprized', `anxienty', `expectation', `anger' and `sereneness'.}
We decide the emotion of the tweet sentence.
Then, we automatically add an emoticon to the sentence.

 %アブストラクトページの作成
 \makeabstract
 \newpage
 
 %目次の生成
 \tableofcontents
 \newpage
 
 %ここからページ数1
 \pagestyle{plain}
 \setcounter{page}{1}
 
 
 %本文
\section{はじめに}\label{sec:begin}
 現在、Social Networking Service (SNS) の普及により多くのユーザーがSNSを利用している。
SNSとは、人と人とのコミュニケーションをインターネットを通してサポートするコミュニティ型の会員制 Web サービスのことである。
これには、チャットやLINE\raisebox{.5zw}{\scalebox{.5}{\textregistered}}のような書き手がやりとりの速さを求めるサービスや、Twitter\raisebox{.5zw}{\scalebox{.5}{\textregistered}}のような短文でやりとりするサービスがある。

このようなインターネットを介してのコミュニケーションにおいては、「書き手の感情を読み取れるテキスト」と、「読み取ることができないテキスト」がある。
例えば、「上司に褒められて嬉しかった」と「後で君に話したいことがある」では、後者の感情を読み取ることが難しい。
このような文から書き手の感情を取得することができれば、そこからコミュニケーションをとる上で得られる情報は大きい。

このように、テキストから感情を抽出することを「感情推定」という。
具体的には、文中に表現されている感情を、自然言語処理により判別することと言える。
この感情推定は、膨大な量のインターネット上の書類を扱うための主要な技術の一つになっている。
例えば、レビュー記事から商品・サービスに対する感情を抽出することにより、消費者の感情情報収集・企業の市場調査が可能となる。
そして、テキスト読み上げロボットの表情・ジェスチャー構成など幅広い応用がある\ibcite{suga}。

感情推定にはいくつかの手法が存在する。
基本的な手法として挙げられるのは、感情語辞書を使用する方法である。
この感情語辞書とは、語彙とその語彙が表す感情を記録したものである。
そこで、この手法では、その感情語辞書を使用して、与えられた文について語彙単位で感情を評価し、その文全体の感情を判定する。

次に挙げられる手法は、感情コーパスを使用する手法である。
感情コーパスとは、その文がどのような感情を持っているかを付与した文の集まりである。
この手法では、感情コーパスを学習データに使い、Support Vector Machine (SVM)やナイーブベイズ分類器を使用し、単語の出現頻度から対象文の感情値を出力する ~\ibcite{yamamoto}。
また、その感情コーパスの作成には、コーパス内の文章をそれぞれ人手でタグ付けする方法や、あらかじめシソーラスや辞書から感情に関係する語を分類して半自動でコーパスを作る方法 ~\ibcite{matsumoto} がある。

感情コーパスを用いた感情推定方法には、ナイーブベイズ分類器を使用したものがある。
そこでは、感情属性がタグ付けされた文より感情語の出現頻度を計算することによって感情推定を行う。
このナイーブベイズ分類器のモデルには、多項モデルと多変数ベルヌーイモデルがある。
多項モデルでは文書の各位置でどの単語が出現したかをモデル化しているのに対し、多変数ベルヌーイモデルは各単語が出現の有無のみをモデル化している。
そのため、テキストの感情推定において、多項モデルは多変数ベルヌーイモデルより性能が高いことが明らかになっている\ibcite{andrew} \ibcite{suzuki}。

ただ一方で、近年では、文末に顔文字を付与して文の感情を表すユーザが多い。
顔文字を付与することで、インターネット上のコミュニケーションを円滑にすることができる。
顔文字は、「\verb|(^_^)|」のように、記号や文字を組み合わせて表情を表現したものである。
このように正位置 \footnote{テキストを読むときの文字の横方向} で顔を表現することは、主に日本で使用される顔文字の特徴であり、これを「東洋式顔文字」と呼ぶ。
顔文字には文の感情を強調、補足できるという利点があるが、顔文字の種類は増大し続けているため、ユーザが文に適切な顔文字をただ1つ選ぶことは難しい。
この問題に対して、江村らはユーザの入力文から感情を推定し、顔文字を推薦するシステムを提案した\ibcite{emura}。
しかしながら、この江村らの手法では、入力文の感情推定に多変数ベルヌーイモデルに類似した手法が用いられており、その感情推定が十分であるとは言えない。

本研究の目的は、単文化しがちなSNS上のコミュニケーションに対して、感情の受け渡しを補助するために、入力文に対して顔文字を自動的に付与することである。
よって、感情推定のために、まず感情コーパスを用いて多項モデルでナイーブベイズ分類を行う。
それにより、顔文字付き感情語辞書を構築する。
そして、構築した辞書をもとに、入力文の感情をナイーブベイズ分類により推定し、推定した感情に適切な顔文字を推薦する。
確率付き感情語辞書には、感情語と、その感情属性、感情確率を登録する。
感情確率とは、文中にある形態素が出現した際に、その文がある感情であると判断される確率である。
この感情確率の計算には、ナイーブベイズ分類を用いた。
舟根の感情コーパスを教師データとして、Twitter から収集した Tweet 約12万件をナイーブベイズ分類し、確率付き感情語辞書を構築した。
そして、この確率付き感情語辞書をもとに、顔文字付き確率辞書を構築した。
顔文字のデータとして、Twitter の 顔文字付き Tweet を使用した。
まず、Tweet 文と顔文字を分割した。
次に、Tweet 文を形態素解析し、確率付き感情語辞書を参照して、最も感情確率が大きい形態素を選び出した。
そして、最も感情確率が大きい形態素の感情を、文の感情とした。
また、その文に付与されていた顔文字に、文と同じ感情を付与した。
そして、顔文字と、感情を顔文字付き感情確率辞書に登録した。
このとき、顔文字の、それぞれの感情に属する確率を求め、7感情すべての感情の感情確率も登録した。
7感情すべての感情確率を、感情確率ベクトルと呼ぶ。
最後に、入力文を形態素解析して、確率付き感情語辞書を参照して入力文の感情を推定する。
そして、推定した感情に適切な顔文字を、顔文字付き感情確率辞書から選び、付与する。
このとき、入力文の感情確率ベクトルと顔文字付き感情確率辞書の感情確率ベクトルのハミング距離をとり、最もハミング距離が小さい顔文字を入力文に付与する。
ハミング距離が小さければ、より入力文に適切な顔文字であると判断した。

本論文の構成は以下の通りである。
\sref{sec:begin} では、本研究の背景及び目的を記述した。\sref{sec:relatedworks} では、本研究を行うにあたって参考にした文献・研究を説明する。
\sref{sec:def} では、本研究で構築した確率付き感情語辞書と顔文字付き感情確率辞書の定義と、これらを構築する際に使用したナイーブベイズ分類器ついて説明する。
\sref{sec:experiment}では、精度検証実験の手順とその結果を記述する。
\sref{sec:analyze}では、\sref{sec:experiment} で行った実験結果をもとに考察を行う。
\sref{sec:summary}では、本研究の成果と今後の展望を述べる。

\section{感情抽出に関する研究}\label{sec:relatedworks}
人間は、日常的に日本語や英語、手話などの自然言語を用いて意思疎通を行う。
自然言語とは、音素、形態素、語、文、文章という階層構造を持っている。
自然言語は、人間の意思疎通手段として自然発生的に構築されたため、人間は自然言語を直感的に理解できる。
しかし、コンピュータで処理を行うためには、自然言語処理を行う必要がある。
この自然言語処理とは、自然言語を機械的に処理するための要素技術およびその応用を含む技術分野である。

本研究では、この自然言語処理の技術を用いて文の感情抽出を行う。
感情抽出とは、テキストの書き手の感情を判別する研究である。
感情抽出に関する研究には、感情コーパスを用いて文単位に感情を抽出する方法や、感情語辞書を用いて品詞単位に感情を抽出し、形態素解析を行って感情のスコアを算出する方法がある。
感情コーパスとは、文章にその書き手の感情を表す感情タグを付与したコーパスである。
感情コーパス中の文と入力文の類似度を算出することによって感情抽出ができる。
また、感情語辞書とは、形容詞や動詞などの一般品詞に感情の尺度を付与した辞書である。
そして、形態素解析とは、文を形態素と呼ばれる意味のある最小単位に分割し、辞書を利用してそれぞれの品詞や内容を判別することである。
形態素解析はコンピュータによる自然言語処理の基礎技術の一つであり、かな漢字変換や機械翻訳、音声認識で使用される。
入力文を形態素解析し、感情語辞書を参照して入力文の感情を抽出することができる。

感情抽出の従来研究では、単語に感情をタグ付けしてまとめた感情語辞書をもとにテキストから感情を推定するものが多くある。
感情を推定する基準となる感情語については、生起事象文型パターン \ibcite{matsu} に基づいた会話文からの感情推定方法で定義されている。
この感情語とは、単体で感情を表現することが可能な単語である。
例えば、名詞であれば「楽しみ」や「がっかり」、動詞であれば「驚いた」や「怒る」などである。
感情語にはこのような感情の種別を表す感情属性が与えられている。

\begin{comment}
感情属性のほかに感情属性値と感情パラメータが付与されている。
この感情属性値とは、単語がその感情属性にどれだけ帰属するかの度合いを表す。
感情パラメータとは、文中の各要素に付与された感情属性値を感情の種類ごとに合成したものを指す。
感情パラメータは、$(各感情属性の数) * (文中の格要素数 +1)$ の感情ベクトルで表現する。
\end{comment}

感情コーパスによる感情抽出の手法には、菅原の研究がある\ibcite{suga}。
これは、まずコーパスを学習データとして、SVMやナイーブベイズ分類器を使用する。
それにより、感情がタグ付けされた文から単語の出現頻度を算出する。
これにより対象文の感情を出力している。

感情コーパスの作成には、松本らが提案したコーパス内の文章をそれぞれ人手でタグ付けする方法や、シソーラスや辞書から感情に関係する語を分類して半自動でコーパスを作る方法がある。
また、舟根による、テキストをuni-gram、bi-gram、tri-gram の三つの分割方法で感情推定し、感情が全て一致した文と感情を感情コーパスに追加することで感情コーパスを半自動的に構築する研究もある ~\ibcite{funane}。

本研究では、テキストを形態素解析し、それぞれの形態素に感情属性と感情属性値を与えて感情語辞書を作成する。
この感情語辞書は、舟根による研究で半自動生成されたコーパスを元に作成する。
このとき、感情語の感情属性値はナイーブベイズ推定によって算出する。
そして、構築した感情語辞書をもとにして文の感情を推定する。

\subsection{顔文字解析の研究}\label{sec:kaomojichushutsu}
近年では、SNS において書き手の感情を表現するために顔文字が使用されている。
この顔文字は、括弧を顔の輪郭とし、横方向に数学記号や漢字、ひらがな、カタカナを用いて表情を表現する。

顔文字には、文の感情を強調する機能や、文の感情を和らげる機能があることが分かっている\ibcite{kawakamikao}\ibcite{kawanokao}。
しかし、顔文字の種類は増加し続けており、文の書き手がその中からただ1つの顔文字を選ぶことは時間がかかる。
また、顔文字入力の補助となる顔文字辞書や予測変換機能は、感情以外での目的の顔文字を選ぶことが難しい。
そのほかの手段として、Web 上にある顔文字辞書からコピーアンドペーストする方法がある。
しかし、この方法では手間が多くかかり、非効率的であるといえる。

この問題を解決するため、顔文字からの感情抽出や、文への顔文字推薦システムを提案する研究がされている。
篠山らは、顔文字を考慮した対話テキストの感情推定に関する研究 \ibcite{shinoyama} を行った。
この研究では、顔文字を考慮した感情推定システムが提案された。
この感情推定システムでは、文章と顔文字それぞれの感情を推定し、両方の結果をもとに全体の感情を推定している。
しかし、この研究では、対象とする顔文字が少ないため、幅広い顔文字推薦への応用は難しい。
そこで、より大きい規模のものとして、Ptaszynski らの CAO システム \ibcite{cao}があげられる。
このシステムでは、顔文字をパーツごとに分け、感情をパーツごとにタグ付けすることで、顔文字からの感情抽出を行った。
その結果、CAO システム全体の精度は 90\% となった。

また、顔文字解析に関連して、橋本 \ibcite{hashimoto} の Twitter の文に絵文字を推薦する研究がある。
この研究では、まず入力文を形態素解析し、それを単語 3-gram に分割する。
そして、絵文字入りコーパスを用い、単語 3-gramと類似するコーパス中の文で用いられている絵文字を文章に挿入する。
また、ポジネガ推定を行ってから感情推定の 2 段階推定にすることで、精度が向上することが徳久らの研究\ibcite{tokuhisa} から明らかになっている。

本研究では、Twitter から 顔文字付きの Tweet 文を集める。
そして、感情コーパスをもとに、Tweet 文に含まれる感情語を手がかりとして、感情コーパスから類似文を見つける。
最も類似度が高い感情をその Tweet の感情とする。
その際、Tweet に付与されていた顔文字にも同じ感情をタグ付けする。
この作業を繰り返し自動で行うことによって、顔文字辞書を構築する。
この顔文字辞書を用いることで、感情語を含む入力文に対し、顔文字を推薦するシステムを提案する。

\section{ベイズ推定と感情推定}\label{sec:def}
本研究では、文の感情感情推定についてベイズ推定を用いている。
そこで、\ssref{sec:defbayes}では、ベイズ推定とベイズの定理を説明する。
\ssref{sec:defnaivebayes}では、本研究で用いたナイーブベイズ分類について説明する。

  \subsection{ベイズ推定とは}\label{sec:defbayes}
ベイズ推定とは、ベイズ確率に基づき観測事象から推定したい事象を確率的に推論する手法である。
ベイズ推定の基本的な方法論となっているベイズの定理は、18世紀にイギリスの確率論研究家トーマス・ベイズが提案した。
このベイズ確率は、ベイズの定理を用いて複数の命題の各々のもっともらしさを確率値で表したものと言える。
このベイズの定理は統計学に応用され、ベイズ統計学の代表的な方法となっている。


ベイズ推定には膨大な反復計算が必要となる。
なぜならば、ベイズの定理においては、事前確率を求めるための事象が増える度にその確率値を更新する必要があるからである。
しかし、コンピュータの処理性能が向上したことにより、現在は心理学や経済学、そしてインターネットの検索やメールフィルタに利用できるようになっている。

そのベイズの定理は、以下の等式で表される。

\[
P(B|A)=\frac{P(A \cap B)}{P(A)}=\frac{P(A|B)P(B)}{P(A)}
\]


$P(A)$、$P(B)$は事象$A$、$B$が発生する確率である。
$P(B)$は$P(A)$が起きる前の事象$B$の確率であり、これは事前確率と呼ばれる。
$P(B|A)$は事象$A$が発生したあとに事象$B$が発生する条件付き確率を表す。これを$B$事後確率と呼ぶ。
また、$P(A|B)$は事象$B$が発生したあとの事象$A$が発生する条件付き確率を表す。
%WIP
これは$B$が真の際にデータが得られる尤度 \footnote{} を表し、尤度関数と呼ばれる。
$P(A \cap B)$は事象$A$が発生し、事象$B$も発生する確率を表す。
この式から、事後確率$P(B|A)$は事前確率$P(B)$と尤度$P(A|B)$の積に比例することが分かる。

\begin{equation}
P(B|A)=\frac{P(A|B)P(B)}{P(A)}
\label{shiki1}
\end{equation}

において、両辺に$P(A)$を掛けると

\begin{equation}
  P(A) \cdot P(B|A)=P(B) \cdot P(A|B) \label{shiki2}
\end{equation}

となる。

また、全体集合を$U$としたとき、各確率$P(A)$、$P(B)$は以下のように書くことができる。

\begin{eqnarray*}
P(A) = \frac{A}{U}\\
P(B) = \frac{B}{U}
\end{eqnarray*}

つまり、これは全事象において各目的事象$A$、$B$が現れる確率を表す。
そこで、以下の条件付き確率を考える。
ここで出現する条件確率は$P(A|B)$、$P(B|A)$で、その定義により次のようになる。

\begin{eqnarray*}
P(A|B) & = & \frac{A \cap B}{B}\\
P(B|A) & = & \frac{B \cap A}{A}
\end{eqnarray*}

したがって、これを式(\ref{shiki2})の両辺に当てはめると次のようにできる。

\begin{eqnarray*}
P(A) \cdot P(B|A) & = & \frac{A}{U} \cdot \frac{A \cap B}{A}\\
               & = & \frac{A \cap B}{U}\\
P(B) \cdot P(A|B)  & = &\frac{B}{U} \cdot \frac{B \cap A}{B}\\
                  & = & \frac{B \cap A}{B}
\end{eqnarray*}

つまり、

\[ P(A) \cdot P(B|A)=P(B) \cdot P(A|B) \]

となり、式(\ref{shiki1})が成り立つことが示される。
このように、ベイズ定理では、ある事象Aが発生したときに、その原因事象が事象Bであった確率を、求めることができる。つまり、ある入力文があったとき、その感情が何であったのかという確率を事前に与える文と、その感情で計算できることを示す。
また、ベイズ推定には、サンプル数が少ない場合やモデルが複雑である場合にも推定が可能であるというメリットがある。

 \subsection{ナイーブベイズ分類}\label{sec:defnaivebayes}
ナイーブベイズ分類は、ベイズ推定の一種であり、過去の事例をもとに未知の文書があらかじめ与えられている場合、それぞれの文書が独立に生起していると仮定した上で、それがどのカテゴリに属するかを決定する分類手法である。
簡単には、未知の文書に対して、事後確率が最大となるクラスを出力することで分類を行う。

このナイーブベイズ分類においては、単語間の相関関係を考慮せず、単語の出現確率のみを学習させればよいので、実装が容易で、さらに学習時間も短いことが特徴として挙げられる。
なお、有効な特徴を用いることができれば、高い分類精度が得られる。

次に、ナイーブベイズ分類の数式について説明する。
ナイーブベイズは、ベイズの定理の数式を用いて分類を行う。
まず、コーパスの1文を文$s$とみなす。
文$s$はいくつかの形態素$w_1,w_2, \cdots $で構成されている。
また、このとき各感情カテゴリを$c_1,c_2 \cdots ,c_i$とする。
すると、ある文 $s_j$ があるとき、その$s_j$の感情が$c_i$である確率は次のように表される。
事後確率$P(c_i|s_j)$は、以下の数式で求められる。

\[
P(c_i|s_j)=\frac{P(c_i|w_1,w_2, \cdots)}{P(c_i)}=\frac{P(w_1,w_2, \cdots|c_i)P(c_i)}{P(c_i)} \propto P(w_1,w_2, \cdots|c_i)
\]

このように、上式は$P(w_1,w_2, \cdots|c_i)$に比例するため、$P(c_i)$は無視することができる。
また、ナイーブベイズでは各感情カテゴリのもとで形態素は独立に生起すると仮定するため、感情カテゴリにおける各形態素の出現確率は $P(w_k|c_i)$ で求められる。
よって、$P(w_1,w_2, \cdots |c_i)$はその積で求めることができるため、$P(w_1,w_2, \cdots |c_i)$は以下の式で表される。

\[
P(w_1,w_2, \cdots|c_i)=\prod_{k=1}P(w_k|c_i)
\]


したがって、文$s_j$に対する感情$c_i$の事後確率$P(c_i|s_j)$は以下のように求めることができる。

\[
P(c_i|s_j)=P(c_i)\prod_{k=1}P(w_k|c_i)
\]

このとき、$P(c_i)$ はカテゴリ$c_i$ が得られる確率であり、これは以下のように表すことができる。

\[
P(c_i)=\frac{カテゴリc_i と判定された文の数}{すべてのカテゴリの全文数}
\]

また、ある単語 $w_k$ が感情カテゴリ $c_i$ に出現する確率$P(w_k|c_i)$は以下のように表すことができる

\[
P(w_k|c_i)=\frac{単語 w_k が出現する回数}{感情カテゴリに含まれる全単語数}
\]

その結果、$P(c_i|s_j)$ 文$s_j$ が与えられた時にカテゴリ$c_i$ が得られる確率は、以下のように表すことができる。

\[
P(c_i|s_j)=\frac{カテゴリc_i に属する文の数}{コーパスに含まれる全文数}
\]

これにより、単語$w_j$が出現した際、$w_j$がカテゴリ$c_i$に属する確率を求めることができる。

本研究では、このナイーブベイズ分類を用いて、ある文 $s_i$ が入力された際に、ある顔文字 $f_i$ が出力される確率を求める。
この確率を顔文字確率と呼ぶ。
そして、入力文に対して最も顔文字確率の高い顔文字を推薦する。
顔文字確率は以下の数式で求めることができる。


\[
  P(f_i|s_i)=\frac{P(s_i|f_i) \cdot P(f_i)}{p(s_i)}
\]

このとき、左辺は「ある文 $s_i$ が入力されたときに、ある顔文字 $f_i$ が出力される確率」である。
右辺のベイズの式は、$P(s_i|f_i) \cdot P(f_i)$ に比例するため、$P(s_i)$ は無視することができる。
つまり、以下のように表すことができる。

\[
  P(f_i|s_i)=P(s_i|f_i) \cdot P(f_i)
\]

このとき、$P(s_i|f_i)$は、「ある顔文字 $f_i$ があったときにある文 $s_i$ である確率」である。
しかし、コーパス内に完全に一致する文が存在する確率は低いため、「ある顔文字 $f_i$ があったときにある形態素 $w_i$ である確率」を求める。
例えば、「\verb|(*^o^*)|」という顔文字が出現した際の Tweet 文に「ありがとう」が含まれる確率である。
Tweet 文に対しての形態素は、以下のように示すことができる。

\[
s_i = w_1^i,w_2^i,\ldots,w_n^i,f^i
\]


以上のことをまとめると、以下の式に表すことができる。

\[
  P(f_i|s_i)=P(s_i|f_i) \cdot (f_i) \simeq (w^i_1,w^i_2, \cdots,w^i_n|f_i)\cdot P(f_i)
\]

つまり、$P(w_i|f_i)$は、以下のように表すことができる。
\[
P(w_i|f_i) = P(w^i_1,w^i_2,\ldots,w^i_n|f^i) = \prod_{j=1} P(w^i_j|f^i)
\]

そして、顔文字 $f_i$が出現したときに形態素 $w_i$ は以下のように表す。
\[
P(w^i_j|f^i) = \frac{顔文字 f^i が含まれる全てのTweetの中で形態素 w^i_j が出現する回数}{顔文字 f^i があるtweetの総形態素数}
\]

以上のことから、形態素 $w_i$が出現したときに顔文字$f_i$が出現する確率を以下の式で表すことができる。
\[
  P(f_i|w^i_1,w^i_2, \dots w^i_n) = P(f_i) \cdot P(w^i_1,w^i_2, \dots ,w^i_n|f_i) = P(f^i) \cdot \prod_{j=1} P(w^i_j|f^i)
\]

上記の手法で、入力文に対して顔文字を付与することができる。
しかし、入力文に顔文字を付ける際に、入力文の各形態素に対して、顔文字それぞれの確率を計算する必要がある。
そうすると、$顔文字の種類 \times \mathit{tweet}文に含まれる形態素数$」が毎回計算される。
顔文字の種類は、収集した約12万件の Tweet に 1つ付与されているとして、12万件存在する。
入力文の形態素の平均文字数を3文字と考えると、入力文の平均形態素数は $\frac{\frac{140}{2}}{3}=23$ と考えられる。
そのため、上記の手法を用いると約 $120000 \times 23 \simeq 2760000$ 回の確率計算が必要となる。
これを計算して、一番確率が高い顔文字を選出するには時間がかかってしまう。
そのため、感情語辞書と顔文字辞書を用いて入力文に適切な顔文字を推薦する手法を提案する。

  \subsection{本研究で用いるナイーブベイズ分類}\label{sec:thesisnaivebayes}
本研究では、舟根の研究で生成された感情コーパスをもとに、確率付き感情語辞書を生成する。
その後、Twitter から収集した 12 万件の tweet をもとに、tweet の文中に付与された顔文字を用いて顔文字付き感情確率辞書を生成する。
これらを生成する際、感情属性値を算出するためにナイーブベイズ分類を用いる。

まず、分類単位であるコーパス中の1文を文$s$とする。文$s$はいくつかの単語$w_1,w_2, \cdots ,w_j$で構成されている。
各感情カテゴリを $c_1,c_2 \cdots ,c_7$とすると、$s_j$が分類されるべき感情属性は、$P(c_i|s_j)$を最大化するような感情となる。
これは、ナイーブベイズ分類から以下の式表すことができる。

\begin{eqnarray*}
c & = & {c_1,c_2 \cdots ,c_7}\\
\hat{c} & = & \argmax_{c} P(c_i|s_j) \\
        & = & \argmax_{c} P(c_i|w_1,w_2,\cdots ,w_k) \\
        & = & \argmax_{c} P(w_1,w_2,\cdots ,w_k|c_i)P(c_i) \\
\end{eqnarray*}

文の分類には次式を最大化する感情属性$\hat{c}$を選ぶべきであるため、

\[
\hat{c}=\argmax_{c_i} P(c_i)\prod_{k=1}^n P(w_k|c_i)
\]

のようにして各単語の感情属性値の積となる最大値$\hat{e}$を求める。

$P(w_j|e_i)$は以下のように表すことができる。

\[
P(w_j|e_i)=\frac{感情 e_i とタグ付けされた全ての文で形態素 w_j が出現する回数}{感情 e_i とタグ付けされた全ての文の形態素の数}
\]

これにより、文中の形態素 $w_i$が出現した際に、感情$e_1,e_2, \cdots ,e_7$にそれぞれどのくらい帰属するかを求めることができる。
形態素が感情にどれだけ帰属するかという確率を、感情確率と呼ぶ。
算出した感情確率をもとに、確率付き感情語辞書を構築する。
この確率付き感情語辞書を用いて、入力文と感情コーパスの類似度を算出し、最も類似度が近い感情を入力文の感情とみなす。

\subsubsection{ゼロ頻度問題}\label{sec:zero}
未知の単語$W_\mathit{new}$が出現した場合、$W_\mathit{new}$ の出現確率 $P(W_\mathit{new}|C)$ は0と推定される。
$P(W_\mathit{new}|C)$ は複数の単語例の確率の積から算出する。

以下に数式を示す。

\[
P(W_\mathit{new}|C)=\frac{単語 W_\mathit{new} の数}{カテゴリC の語彙数}
\]

ここで、文中のどれか1つの単語が感情語辞書に登録されていない場合、その単語の感情確率は0となる。
そのため、文全体の確率も0となってしまうという問題が発生する。
この問題を、ゼロ頻度問題と呼ぶ。


この問題を避けるために、$W_\mathit{new}$の出現確率を求める際に、単語の出現回数を補正した値を用いるスムージングが行われる~\ibcite{textmining}。

スムージングは以下の様に行われる。

\[
P(W_\mathit{new}|C)=\frac{単語 W_\mathit{new} の数 + \alpha }{カテゴリC の語彙数 + \alpha }
\]

一定の数を予め加えておくことで、未知の単語$W_\mathit{new}$が出現した場合に確率を0にさせないことができる。

ここでは単純な加算方であるラプラス法を用いてスムージングを行う。
これはラプラススムージングと呼ばれる。

ラプラススムージングでは、加算法において単語の出現頻度を求める際に全ての単語の出現回数に $\alpha=1$ を加える手法を用いる。
 
\section{感情推定における自然言語処理の基礎技術}
\subsection{Twitter API}
 Twitter API とは、Twitter 社が無料で提供する API サービスのことである。
API とは、Application Programming Interface のことであり、あるソフトウェアの機能や管理するデータなどを、外部の他のプログラムから呼び出して利用するための手順やデータ形式のことである。
Twitter API を利用することで、ウェブサイトやアプリなどから Twitter の機能の呼び出しや検索などをプログラム上で扱うことができる。

Twitter API には、REST API と Streaming API の二種類がある。

REST API は、HTTP 接続をその都度行うことで、蓄積された過去の Tweet 情報の取得や Tweet の投稿、ユーザ情報の変更などをプログラム上で行うことができる基本的な API である。
通常の Twitter クライアントや BOT で利用される。
REST API は URL に要求などを定められた形式で文字列にしたクエリを渡し、そのレスポンスを取得する REST(Representational State Transfer) を利用している。
REST API は利用規制があり、一定時間ごとにAPI の呼び出せる回数が制限されている。
そのため、制限数を使い切ってしまった場合、API の呼び出しが一定時間できなくなる。
API 1.1 では15分間にアプリごと・API ごとにそれぞれ呼び出しの制限回数が定められている。

Streaming API は一度接続すると HTTP 接続を保ったまま、接続が切断されるまで Tweet 情報を取得し続けることができる API である。
そのため、タイムラインの変更をリアルタイムに受け取ることができる。
Streaming API はデータ解析などで大量の Tweet 文を必要とする開発者向けの API である。
仕組みとしては、REST API と同じく REST でアクセスを行う。
しかし、タイムラインの更新が発生するまでレスポンスを保留させるとこで、高い瞬時性を実現している。
一度に何本も API を接続することができない点や、切断と接続を短時間に連続で行うことができない点、過去の Tweet 文を取得できないなどの制約がある。
しかし、REST API とは違い、API の利用規制がかからないという利点がある。

本研究では、顔文字や文章の感情推定について、ナイーブベイズを使用している。
この場合、ベイズ推定の推定確率の確からしさを大きくするためには、その確率を計算するためのベースとなるコーパスの量が重要となる。
このコーパスとは、ある目的において無作為に集められた多量なテキストのことを言い、本研究で必要となるコーパスはすなわち、1) 文章より感情推定が容易なテキスト 2) 顔文字が使用されているテキスト、これら二つの条件を充すような大量のテキストである。
このような大量のテキストを取得する SNS には Twitter が適切であると考え、本研究では Streeming API を使用し、リアルタイムに Tweet 文の取得を行った。

Twitter API には以下のような機能がある。

\begin{itemize}
  \item Tweet の投稿
  \item Tweet の検索
  \item Tweet の取得
\end{itemize}

このうち、Tweet を取得する機能を使用する。
本研究では、Steaming API を使用し、リアルタイムに Tweet 文の取得を行った。
その結果、約12万件の Tweet 文を取得した。
取得した Tweet 文の例を以下に示す。

\begin{itembox}[c]{Tweet 文の例}
  こんなに素敵なものを受け取っていいのか\verb|(T_T)|ほんとにありがとう!\\
  おやすみなさい\verb|(´ω`)| \\
  スーパーワーカーってゆう響きが素敵です\verb|( ´ ▽ ` )| \\
  明日朝早いねん\verb|(^_^;)| \\
  あっ、なんかTwitterとかFBで見ました(笑)\\
  やっぱ羨ましくない!拗ねてる\verb|(`へ′)| \\
  痛いしむかつくんだけど\verb|(*_*)| \\
  楽しみです\verb|(*^^*)| \\
  がんばれ〜っ!\verb|ヽ( ´ー`)ノ|\\
\end{itembox}

これらの Tweet を形態素解析し、感情語辞書と顔文字辞書に感情語と顔文字をそれぞれ登録した。
顔文字付きの Tweet を取得する際、括弧に囲まれたすべての文字列を顔文字と認識した。
そのため、取得した 12万件の Tweet 文の中には、「(笑)」や「(月)」など、顔文字以外の文字列が含まれる Tweet が登録されるという問題が発生した。
この問題を回避するため、顔文字辞書に登録する顔文字は、括弧の中に漢字、ひらがな、カタカナ、数字が含まれたもの以外とした。
また、より顔文字辞書の精度を上げるため、異なる Tweet の中で2度以上使用された顔文字を顔文字辞書に登録した。
これにより、顔文字ではない文字列を顔文字辞書から取り除くことができた。


\subsection{MeCab と Natto}
\subsubsection{形態素解析}
本研究では、感情語辞書を構築するために、形態素解析を行った。
形態素解析 (Morphological Analysis) とは、自然言語処理の手法の一つである。
ある文章やフレーズを、意味を持つ最小限の単位である形態素に分解し、辞書を利用して形態素の品詞や活用などの情報を付け加えることで文章やフレーズの意味を判断するために用いられる。
形態素解析は現在、かな漢字変換や機械翻訳、音声認識で使用されている。

例えば、「私は家で勉強します」という日本語の文は、形態素解析を用いると「私」、「は」、「家」、「で」、「勉強」、「し」、「ます」という形態素から成り立っていることが分かる。
この形態素解析を用いて、文からの感情抽出を試みる。

\subsubsection{MeCab と Natto}
本研究では、形態素解析に MeCab と Natto を使用した。
まず、Mecab について述べる。
MeCab \ibcite{mecab} は、奈良先端科学技術大学院大学が開発したオープンソースの形態素解析エンジンである。
MeCab は単体では機能せず、形態素ごとに分割するためには品詞や意味が内蔵された辞書が必要になる。
MeCab 用の基本的な辞書として IPA 辞書や Juman 辞書、Unidic 辞書がある。
これらの基本的な辞書を使うことはもちろん、自分で任意の単語を手動で辞書に追加することも可能である。
また、ユーザが開発した様々なオリジナル辞書も Web 上に存在する。
本研究では、IPA 辞書を使った MeCab で形態素解析を行う。

MeCab を用いて例文「私は家で勉強します」を形態素解析すると、以下のような結果が得られる。

\begin{itembox}[l]
私  名詞,代名詞,一般,*,*,*,私,ワタシ,ワタシ\\
は  助詞,係助詞,*,*,*,*,は,ハ,ワ\\
家  名詞,一般,*,*,*,*,家,イエ,イエ\\
で  助詞,格助詞,一般,*,*,*,で,デ,デ\\
勉強  名詞,サ変接続,*,*,*,*,勉強,ベンキョウ,ベンキョー\\
し  動詞,自立,*,*,サ変・スル,連用形,する,シ,シ\\
ます  助動詞,*,*,*,特殊・マス,基本形,ます,マス,マス\\
EOS
\end{itembox}

上に示したように、MeCab は CSV (Commma Separated Value) 形式になっている。
MeCab の出力フォーマットは、左から、表層形、品詞、品詞細分類1、品詞細分類2、品詞細分類3、活用形、活用型、原形、読み、発音となっている。

また、本研究では、MeCab を利用するために、Natto という Gem パッケージを使用した。
Gem パッケージは、プログラミング言語 Ruby で使用されるライブラリやアプリケーションのパッケージである。
Natto の他にも多くのライブラリが Gem 形式でパッケージされて公開されている。

Natto を用いることで、Ruby で簡単に MeCab を呼び出すことができる。
例えば、「今日は学校を休みたい」という文を MeCab を用いて形態素解析するとき、以下のようにプログラムを書く。

\begin{itembox}[l]{プログラムの例}
require 'natto' \\ \\

text = '今日は学校を休みたい' \\ \\

nm = Natto::MeCab.new \\
nm.parse(text) do |n| \\
  \verb|puts "#{n.surface} #{n.feature}"| \\
end
\end{itembox}

上記のプログラムを実行すると、以下のように表示される。

\begin{itembox}[l]{プログラムの例}
  今日  名詞,副詞可能,*,*,*,*,今日,キョウ,キョー \\
  は  助詞,係助詞,*,*,*,*,は,ハ,ワ\\ 
  学校  名詞,一般,*,*,*,*,学校,ガッコウ,ガッコー\\
  を  助詞,格助詞,一般,*,*,*,を,ヲ,ヲ\\
  休み  動詞,自立,*,*,五段・マ行,連用形,休む,ヤスミ,ヤスミ\\
  たい  助動詞,*,*,*,特殊・タイ,基本形,たい,タイ,タイ\\
 BOS/EOS,*,*,*,*,*,*,*,*\\
\end{itembox}

この Natto を用いることで、自動的に大量の Tweet 文を処理することができる。
また、プログラムから MeCab の形態素結果を操作することができるため、制約付きの解析をすることができる。
つまり、Natto を用いることで、効率的に目的に沿った形態素解析ができると言える。
本研究においては、確率付き感情語辞書に形態素を登録する際、Natto を用いて形態素の原形を登録した。
また、確率付き感情語辞書には、精度を下げないために、品詞が名詞、動詞、形容詞、感動詞である形態素を指定して登録した。
そのため、「私は家で勉強します」という文であれば、「わたし」、「家」、「勉強」、「する」が確率付き感情語辞書に登録される。


\section{顔文字推定システム}\label{sec:theorem}
  \subsection{コーパスを用いた確率付き感情語辞書の作成}\label{sec:tweetcorpus}
本研究では、舟根の感情コーパスを用いて確率付き感情語辞書を作成した。
この感情コーパスの精度は 98\% であった。
これは、N-gram でそれぞれ分割し、uni-gram、bi-gram、tri-gram の感情が一致したものを感情コーパスに登録している。
N-gram とは、あるテキストの総数を前から順に任意の N 個の文字列または単語の組み合わせで分割したものである。
N 個の数 (gram) によって、それぞれ「 uni(1)-gram, bi(2)-gram, tri(3)-gram, $\cdots$ 」と呼ばれる。
本研究では、形態素はそれぞれ独立していると考えるため、uni-gram で感情を抽出し作成した感情コーパスを使用する。
感情語コーパスの例を以下に示す。

\begin{table}[htb]
  \caption{感情コーパスの例}
  \centering
  \begin{tabular}{c|c} \hline
    文 & 感情属性 \\ \hline \hline
    とても楽しい1日だった。 & 喜 \\
    指輪をなくしてしまった & 悲 \\
    この会社、むかつきます。 & 怒 \\ 
    安さに衝撃を受けました。 & 驚 \\
    ちゃんと届くか心配です。 & 不安 \\
    家に届くのが楽しみです。 & 期待 \\
    夫へのプレゼントに購入。& 平静 \\ \hline
  \end{tabular}
\end{table}

まず、感情コーパスを MeCab を用いて形態素解析した。
次に、ナイーブベイズ分類を用いて感情語 $w_j$ が感情属性 $e_i$ で出現する確率 $P(w_j|e_i)$ を算出した。
これを感情確率と呼ぶ。
この際用いた感情語には、動詞、名詞、形容詞、感動詞を用いた。
なお、MeCab を用いて形態素解析した結果の形態素の表層形では活用形が出力されているため、形態素の原型を感情語とした。
感情属性 $e_i$ に感情語 $w_j$ が出現する感情確率 $P(w_j|e_i)$ を求める数式を以下に示す。

\[
P(w_j|e_i)=\frac{感情 e_i とタグ付けされた全ての文で形態素 w_j が出現する回数 +1}{感情 e_i とタグ付けされた全ての文の形態素の数 + コーパスに含まれる形態素数}
\]

最後に、抽出した形態素、感情属性、感情確率を確率付き感情語辞書に登録した。

  \subsection{顔文字付き感情確率辞書の作成}\label{sec:kaomojidic}
Twitter API を用いて収集した tweet をもとに、顔文字付き感情確率辞書を作成した。
顔文字付き感情確率辞書とは、\ssref{sec:tweetcorpus}で作成した確率付き感情語辞書に、tweet の文中に出現する顔文字を付与した辞書である。

まず、収集した tweet を、確率付き感情語辞書を用いて形態素解析した。
tweet 文を $s_i$、それに含まれる形態素を $w^i_k$、顔文字を $f^i$ とする。
この際、tweet の文中に出現する括弧で囲まれている、漢字、ひらがな、カタカナ、数字以外の文字列を顔文字とした。
形態素解析の結果を以下の数式に示す。

\[
s_i=w^i_1,w^i_2,\ldots,w_n^i,f^i
\]

本研究で用いる感情は、`喜'、`悲'、`怒'、`驚'、`期待'、`不安'、`平静'の7感情である。
これらの7感情を$e_1, e_2, \ldots, e_7$ と表す。

tweet 文 $s_i$ の感情推定のため、以下のような表を生成する。

\begin{eqnarray*}
  \begin{array}{r|ccc|l}
        & w^i_1              & \ldots          & w^{i}_{k}    & 感情確率 \\ \hline
    e_1 & P(e_1|w^i_1) &                 & P(e_1|w^i_k) & \prod_{s=1}^{k}P(e_1|w^i_s) \doteq P^i_{e_1}\\
    \vdots             &                 & \ddots       &                                           & \\
    e_7 & P(e_7|w^i_1) &                 & P(e_7|w^i_k) & \prod_{s=1}^{k}P(e_7|w^i_s) \doteq P^i_{e_7}\\
  \end{array}
\end{eqnarray*}


このとき、$P(e_1|w^i_1)$ は 確率付き感情語辞書に登録された感情確率を用いる。
また、$\prod_{s=1}^{k}P(e_1|w^i_s)$ は、感情属性 $e_i$ に対する全感情語の感情確率を乗算した値である。
この値を、顔文字感情確率ベクトルと呼ぶ。
そして、全ての感情属性のうち $\prod_{s=1}^{k}P(e_1|w^i_s)$ の値が最も大きい感情属性を顔文字 $f_i$ の感情属性とする。
最後に、顔文字、推定された感情属性、その感情属性の感情確率、顔文字感情確率ベクトルを、以下のように顔文字付き顔文字辞書に登録する。

\begin{eqnarray*}
  \left\{
   \begin{array}{c}
     f^i, e_s, P^i_{e_s}, (P^i_{e_1}, P^i_{e_2}, \ldots, P^i_{e_7})\\
     \vdots \\
     f^j, e_{s'}, P^j_{e_{s'}}, (P^j_{e_1}, P^j_{e_2}, \ldots, P^j_{e_7})\\
   \end{array}
  \right\}
\end{eqnarray*}


これをもとに作成した顔文字付き感情確率辞書の例を以下に示す。

\begin{itembox}[l]{顔文字付き感情確率辞書の例}
\verb|(^○^)|,喜,6.074e-116,6.074e-116,3.0684e-118,1.376e-119,1.493e-118,1.380e-118,9.310e-120,6.543e-125\\
\verb|(;▽;)|,悲,2.093e-182,8.68e-185,2.093e-182,6.725e-187,1.920e-186,2.223e-186,1.52e-185,1.587e-202\\
\verb|(´・ω・`)|,不安,4.031e-08,3.044e-08,3.472e-08,1.731e-08,1.731e-08,2.699e-08,4.031e-08,1.731e-08\\
\end{itembox}

\subsection{入力文の感情推定}
\ssref{sec:tweetcorpus}で作成した確率付き感情語辞書と、\ssref{sec:kaomojidic}で作成した顔文字付き確率感情語辞書を用いて入力文の感情推定を行い、その文の感情に適切な顔文字を文末に付与するシステムを構築した。
このシステムを顔文字推薦システムと呼ぶ。

顔文字推薦システムの流れを以下に示す。

\begin{itemize}
  \item
    入力文を形態素解析する。
  \item
    出力された形態素(動詞、名詞、形容詞、感動詞)のうち、最も感情確率が高い形態素を、確率付き感情語辞書を参照して取り出す。
  \item
    確率付き感情語辞書と顔文字付き確率感情語辞書の感情ベクトルのハミング距離を計算する。
  \item
    ハミング距離が最も小さい顔文字を入力文の文末に付与する。この際、ハミング距離が同等の顔文字が複数個あった場合には、顔文字をランダムに取り出す。
\end{itemize}

入力文から出力された感情に対応する顔文字を選定する際、顔文字辞書を参照する。
しかし、この顔文字辞書には大量の顔文字が登録されているため、最も適切な1つを選ぶことは難しい。
また、ランダムに1つの顔文字を選び出せば、顔文字推薦の精度が低下するという問題が発生する。

そのため、顔文字辞書の中で最もツイートに適切な顔文字を選ぶ方法として、ハミング距離 (Hamming distance) を用いた手法を提案する。
ハミング距離とは、任意の2つの値を比較したときに値が異なっているビット数の割合である。
また、ハミング距離は、ある文字列を別の文字列に変形する際に必要な置換回数を計測したものである。
1011101 と 1001001 という文字列があった場合、これらの文字列のハミング距離は 2 である。

本研究では、確率付き感情語辞書を用いて入力文の形態素の感情ベクトルを算出し、顔文字付き確率辞書中の感情ベクトルとのハミング距離が最も小さい顔文字を入力文の文末に付与する。
ハミング距離が小さいほど、入力文に対して適切な顔文字を推薦できると考えた。
$f$ を顔文字、$a$ を顔文字付き確率辞書の感情ベクトル、$b$ を入力文の感情ベクトルとしたとき、ハミング距離は以下のように計算する。

\[
  f=\sum_{i=1}^{n} |a_i-b_i|
\]

例えば入力文が「映画最高だった!」である場合、出力される感情は`喜'である。
この文のうち、最も感情確率が高い形態素は「最高」であり、感情確率の値は0.001598であった。
次に顔文字辞書を参照し、この値と最もハミング距離が小さい顔文字を選ぶ。
このとき、ハミング距離の値が同じ顔文字が複数個あった場合にはランダムに顔文字を選ぶ。
そして、入力文の文末に顔文字を付与する。
これを「顔文字推薦」と呼ぶ。
この例文を入力した際、推薦された顔文字は「\verb|(*^o^*)|」であった。

  \subsection{考察}
舟根の感情コーパスから顔文字付き確率辞書をした。その結果、顔文字に感情属性と感情確率を付与することができた。
しかし、"\verb|(´・ω・`)|,喜" や、"\verb|(^ω^)|,悲"など、顔文字に適切ではない感情属性が付与されていることがある。
この原因は、書き手の故意によって、tweet 文中の感情と顔文字が一致させられていないからだと考える。

\section{実験}\label{sec:experiment}
   %手順と条件の説明
本研究で構築した顔文字推薦システムの精度を検証するため、被験者に感情が生起している文章を入力してもらった。
これを被験者5名に1つの感情カテゴリについて5文ずつ、7感情分を入力してもらい、計175文で実験を行った。
入力文に対して推薦された顔文字が適切であれば正解、適切でなければ不正解とし、正解率を求めた。
この正解率を感情推薦システムの精度とする。
精度の計算は、各感情カテゴリごとに、入力文に対して出力された顔文字が適切であると判断された割合の乗数を求めた。
さらに、全カテゴリ分の乗数を掛け合わせ、顔文字推薦システムの精度とした。
各入力文とその正解感情の例を以下に示す。

\begin{itembox}[l]{正解感情付きの入力文の例}
昨日のラーメンがおいしかった。 喜 \\
高級豆安く手に入って嬉しい! 喜\\
チャーシューが柔らかかった。 喜\\
この年になると二日酔いが辛い。 悲\\
この服はあまりかわいくない。 悲\\
カフェのランチじゃ物足りない。 悲\\
届いた靴が意外と大きかった…。 悲\\
東京の物件どれも高い。 怒\\
高速バスは危ない! 怒\\
安い物件はよくないのばかり。 怒\\
卒論がめんどくさい! 怒\\
Amazon Prime Now で注文したら本当に1時間で届いた! 驚\\
電車が5分に一回はくるのがすごい! 驚\\
早く春になってほしい! 期待\\
バーベキューしたい。 期待\\
在庫が残り少なくなってる。 不安\\
ズボンが入りづらい。 不安\\
雪が降ると寒いね。 平静\\
おはよう。 平静\\
めっちゃねむい。 平静\\
\end{itembox}

  \subsection{実験結果}
精度検証実験の結果、システムの精度は 84.0\%であった。
本研究の顔文字出力システムを用いた出力結果を以下に示す。

\begin{itembox}[l]{出力された顔文字と感情の例}
  昨日のラーメンがおいしかった。 \verb|(*´∀`*), 喜| \\
高級豆安く手に入って嬉しい! \verb|( *´v`* ), 喜| \\
チャーシューが柔らかかった。\verb|(*´∀`*), 喜|\\
この年になると二日酔いが辛い。\verb|(´・_・`), 不安|\\
この服はあまりかわいくない。\verb|('∀'), 悲|\\
カフェのランチじゃ物足りない。\verb|(・_・), 悲|\\
届いた靴が意外と大きかった…。\verb|('A`), 悲|\\
東京の物件どれも高い。\verb|( ・ω・), 悲|\\
高速バスは危ない!\verb|(T ^ T), 怒|\\
安い物件はよくないのばかり。\verb|(´・ω・`), 不安|\\
卒論がめんどくさい!\verb|( 'ω' ), 怒|\\
Amazon Prime Now で注文したら本当に1時間で届いた! \verb|(*^o^*), 喜|\\
電車が5分に一回はくるのがすごい!\verb|(@⌒ー⌒@), 喜|\\
早く春になってほしい! \verb|(o^∀^o), 期待|\\
バーベキューしたい。\verb|(つд⊂), 期待|\\
在庫が残り少なくなってる。 \verb|(@ ̄∀ ̄), 不安|\\
ズボンが入りづらい。\verb|(^_^;), 悲|\\
雪が降ると寒いね。\verb|(^_^), 平静|\\
おはよう。\verb|(*´∀`), 平静|\\
めっちゃねむい。\verb|(´・皿・`), 平静|\\
\end{itembox}

また、各感情カテゴリの正解率を以下に示す。

\begin{table}
  \caption{各感情カテゴリの正解率}
  \centering
  \begin{tabular}{|c|c|} \hline
    感情カテゴリ & 正解率 \\ \hline \hline
    喜 & 96\% \\ \hline
    悲 & 92\% \\  \hline
    怒 & 68\% \\ \hline
    驚 & 64\% \\ \hline
    不安 & 80\% \\ \hline
    期待 & 92\% \\ \hline
    平静 & 96\% \\ \hline
  \end{tabular}
\end{table}

このうち、`喜'と、`平静'の感情カテゴリの正解率が最も高く、96\%であった。
しかし、最も正解率の低い`驚'の感情カテゴリは、正解率が64\%であった。
`喜'カテゴリと`平静'カテゴリは、登録される顔文字が似通っている。
`平静'カテゴリには、感情が生起しない文の他に、あいさつをする文や、動作について述べる文が分類される。
そのため、`平静'の文の文末に顔文字を付与することで、顔文字の役割の一つである「画面を装飾する」ことが可能となる。

`驚'の感情カテゴリの正解率が低かった原因は、Twitter への `驚' の感情の Tweet が少なかったことが原因であると考えた。
本研究では、Tweet を元に確率付き感情確率辞書と、顔文字付き確率感情後辞書を用いて文の感情を抽出し、顔文字を推薦する。
そのため、各感情カテゴリに属する Tweet に偏りがあると、顔文字推薦システムに偏りが生まれる。

`悲'と`不安'では、顔文字が似通っていると考えられる。
例えば、「\verb|(;_;)|」という顔文字は、「とってもショックでした。」という`悲'カテゴリの文と、「結果出るまで不安だな。」という`不安'カテゴリの文のどちらにも文末に付与することが自然であると判断される。
そのため、`不安'の感情カテゴリである文に対して`悲'の感情カテゴリの顔文字が付与された場合に正解とするかは実験協力者の価値観に委ねることとした。

\subsection{\sref{sec:experiment}まとめ}
本章では、構築した顔文字推薦システムの精度検証実験を行った。
実験では、5名の被験者に、7感情、全175文を入力してもらった。
そして、推薦された顔文字が適切であれば正解、適切でなければ不正解とし、顔文字推薦システムの正解率を算出した。
この結果、顔文字推薦システムの精度は84\%であることが分かった。

\section{考察}\label{sec:analyze}
\sref{sec:experiment}の精度検証実験の結果、本研究で構築した顔文字推薦システムの精度は 84\%であった。
本研究で確率付き感情語辞書の構築に使用した舟根の感情コーパスの精度は 98\%であった。
顔文字推薦システムの精度が既存研究より下がった原因として、4つ問題点が挙げられる。
以下に問題点を示す。

\subsection{問題点}
問題点の1つは、感情と顔文字が対応していないという問題である。
本研究では、確率付き感情語辞書を用いて文の感情を推定し、顔文字付き確率感情語辞書を用いて推定した感情に適切な顔文字を入力文に付与する。
しかし、顔文字付き確率辞書に登録された顔文字が、感情に適切ではないことが原因で、不適切な顔文字を入力文に付与してしまうことがある。
適切な顔文字の付与に失敗した例を以下に示す。

\begin{table}[htb]
  \centering
  \begin{tabular}{c||c|c} \hline
    入力文 & 推定された感情 & 推薦された顔文字 \\ \hline
    インターネットが繋がらない & 悲 & \verb|( ^ω^)| \\
    本当に腹がたつ & 怒 & \verb|(*´・ω・)|   \\
    明日が楽しみ & 期待 & \verb|(^o^;)| \\ \hline
  \end{tabular}
\end{table}

例えば、「インターネットが繋がらない」という文を形態素解析し、感情語辞書を参照すると、「繋がる」と「ない」の感情確率を取り出す。
これらの感情確率から最も値が大きいものを取り出すと、この文の感情は`悲'となる。
しかし、推薦された顔文字は「\verb|( ^ω^)|」であった。
この顔文字は、目のパーツから推測すると、`喜'の感情に分類されると考えられる。
このように、`悲'の感情属性の文に`喜'の顔文字を付与することは、顔文字推薦システムの精度が低まる要因となる。

この問題を解決するために、高島 \ibcite{takashima} の研究を参考に、顔文字解析を行う必要があると考えた。
高島は、Ptaszynski らによって構築された顔文字解析システム CAO (a system for emotiCon Analysis and decOding of affection information) \ibcite{cao}のアルゴリズムを参考にし、顔文字の感情推定を行った。
このシステムでは、顔文字の目や口のパーツに着目して感情推定を行い、「喜び」、「悲しみ」、「怒り」、「驚き」の4感情ごとに顔文字をデータベースに登録した。
本研究においても、CAO のアルゴリズムを参考にし、顔文字の目や口のパーツやその位置に着目し、顔文字それぞれの感情を推定する必要がある。
そして、ナイーブベイズ分類を用いて顔文字がある感情に属する確率を算出することで、感情に不適切な顔文字が推薦される問題は少なくなると考える。


問題点の2つめは、若者言葉が出現した際の感情抽出の失敗についてである。
現在、Twitter をはじめとする SNS では「やばい」や「ウケる」など、多くの若者言葉が用いられている。
しかし、感情語辞書の元となった感情コーパスは amazon の商品レビューを元に作られており、ここでは年齢層の違いから若者言葉が用いられていない。
そのため、感情を表す形態素が若者言葉であった場合に、若者言葉は未知語となり、文の感情を正しく推定できないという問題が発生する。
例えば、「窓からレインボーブリッジ見えてまじやばい!」という文がある。
これは、「窓からレインボーブリッジが見えることに感動している」という文である。
しかし、感情を表していると言える「やばい」という単語は感情語辞書に登録されていないため、この例文の感情は無感情となる。

この問題に対して、松本らの感情推定における若者言葉の影響 \ibcite{wakamono} の研究が行われた。
この研究では、Weblog から若者言葉を含む文を自動収集し、手動で若者感情コーパスを構築した。
そして、MeCab を用いて形態素解析し、感情語辞書を用いて抽出した単語に感情をタグ付けした。
このとき、文中に現れる顔文字も抽出し、顔文字辞書を用いて顔文字に感情をタグ付けした。
最後に、感情語と顔文字を SVM とナイーブベイズ分類を用いて学習させた。
本研究でも、この研究を参考にし、若者言葉を含む感情コーパスを構築する必要がある。

問題の3つめは、感情語辞書の偏りである。
全ての感情カテゴリの中で、`喜'と`平静'の正解率が最も高く、96\%であった。
しかし、`驚'の正解率は最も低く、64\%であった。
このように、正解率に大きく差が出た原因は、顔文字付き確率感情辞書の登録語数にあると考える。
顔文字付き感情語辞書の全登録語数は 69700語 である。
各カテゴリの登録語数を以下に示す。

\begin{table}[ht]
  \caption{各カテゴリの登録語数}
\centering
\begin{tabular}{c|c} \hline
  感情カテゴリ & 登録語数 \\ \hline \hline
  喜 & 26942 \\ \hline
  悲 & 11165 \\ \hline
  怒 & 6755 \\ \hline
  驚 & 3153 \\ \hline
  不安 & 3077 \\ \hline
  期待 & 5091 \\ \hline
  平静 & 13513 \\ \hline
\end{tabular}
\end{table}

このように、`喜'の登録語数と、`驚'の登録語数には大きく差がある。
各感情カテゴリの登録語数のばらつきを減らすことで、精度を高めることができると考えられる。

問題の4つめは、形態素の前後の関係を考慮していないことである。
例えば、「彼がイライラしていて少し怖い」という文がある。
この文を形態素解析すると以下のようになる。

\begin{itembox}[l]
  彼  名詞,代名詞,一般,*,*,*,彼,カレ,カレ\\
  が  助詞,格助詞,一般,*,*,*,が,ガ,ガ\\
  イライラ  副詞,助詞類接続,*,*,*,*,イライラ,イライラ,イライラ\\
  し  動詞,自立,*,*,サ変・スル,連用形,する,シ,シ\\
  て  助詞,接続助詞,*,*,*,*,て,テ,テ\\
  い  動詞,非自立,*,*,一段,連用形,いる,イ,イ\\
  て  助詞,接続助詞,*,*,*,*,て,テ,テ\\
  少し  副詞,助詞類接続,*,*,*,*,少し,スコシ,スコシ\\
  怖い  形容詞,自立,*,*,形容詞・アウオ段,基本形,怖い,コワイ,コワイ\\
  EOS
\end{itembox}

このとき、感情属性は `怒' であった。
この原因として、「イライラ」という単語の感情確率が大きかったことが考えられる。
この問題を解決するために、Cabocha \ibcite{cabocha}を用いた係り受け解析がある。

係り受け解析とは、文法規則によって、文の構造を句・文節を単位として解析することである。
句とは、2つ以上の語が集まって1つの品詞と同じような働きをしながら、文を構成する語の塊のことである。
名詞の役割を果たす句を名詞句 (NP)、動詞の役割を果たす句を動詞句 (VP) とするように、形容詞句(ADJP)、副詞句(ADVP)などがある。
英語では句構造で構文解析を行うが、日本語の場合は、文節を単位に係り受け関係を用いて構文を解析するのが一般的である。
文節とは、日本語を意味の分かる単位で区切ったものである。
日本語においては、文における任意の1つの文節は少なくともその文節の後の1つの文節と係り受け関係を持つ特徴がある。

係り受け解析の例を以下に示す。

\begin{itembox}[l]{係り受け解析結果}
  * 0 4D 0/1 -2.525582\\
  彼 名詞, 代名詞, 一般, *, *, *, 彼, カレ, カレ \\
  は 助詞, 係助詞, *, *, *, *, *, は, ハ, ワ \\
  * 1 4D 1/2 -2.525582\\
  試合 名詞, サ変接続, *, *, *, *, 試合, シアイ, シアイ \\
  中 名詞, 接尾, 副詞可能, *, *, *, 中, チュウ, チュー \\
  に 助詞, 格助詞, 一般, *, *, *, に, ニ, ニ \\
  * 2 3D 0/0 1.447781\\
  硬い 形容詞, 自立, *, *, 形容詞・アウオ段, 基本形, 硬い, カタイ, カタイ \\
  * 3 4D 0/1 -2.525582\\
  表情 名詞, 一般, *, *, *, *, 表情, ヒョウジョウ, ヒョウジョー \\
  を 助詞, 格助詞, 一般, *, *, *, を, ヲ, ヲ \\
  * 4 -1D 0/1 0.000000\\
  見せ 動詞, 自立, *, *, 一段, 連用形, 見せる, ミセ, ミセ\\
  た 助動詞, *, *, *, 特殊・タ, 基本形, た, タ, タ\\
  。名詞, サ変接続, *, *, *, *, * \\
  EOS \\
\end{itembox}

CaboCha では、形態素解析と共に文節の係り受け関係を解析している。
* から始まる業には文節の係り受け関係が示されており、形態素は行ごとに区切られている。
* の右隣の数字は文節番号を表し、その右横にはその文節が係る文節番号が示されている。
係る文節番号が -1D であった場合、その文節はどの文節にも係っていないことを示す。
この係り受け解析を利用することで、`硬い' が `表情' に係っていることが分かるため、感情属性を付与することができると考えられる。
また、係り受け解析を用いれば文の述部が分かるので、述部の感情に重み付けすることができる。

\subsection{\sref{sec:analyze} まとめ}
本章では、\sref{sec:experiment} で行った精度検証実験を元に、考察を行った。
精度検証実験の結果、本研究で構築した顔文字推薦システムには4つの問題点があることが明らかになった。
問題の1つめは、感情と顔文字が対応していないという問題である。
この問題を解決するために、CAO システムを用いて顔文字から感情推定を行う必要があることが分かった。
2つめは、感情抽出の際に若者言葉を考慮していないという問題である。
言葉はどんどん変化していくものであり、また、Twitter は若年層が多く利用するため、言葉の変化が著しく現れると考える。
そこで、MeCab の辞書に新たに若者言葉を登録し、若者言葉からも感情推定を可能にする必要がある。
3つめは、感情語辞書において各感情カテゴリの登録語数のばらつきによる正解率の低下の問題である。
例えば、`喜'の感情カテゴリの登録語数は26942語であり、`驚'の感情カテゴリの登録語数は3153語であった。
また、精度検証実験において、`喜'の感情カテゴリの正解率は96\%、`驚'の感情カテゴリの正解率は64\%であった。
そのため、2つの感情カテゴリの顔文字推薦の精度に大きく差が出たのは、登録語数に大きな差があるからではないかと考えた。
そこで、顔文字辞書の登録語数を全カテゴリで均等に揃える必要があると考えた。
4つめは、形態素の前後の関係を考慮していないという問題である。
本研究では、形態素全てが独立に生起すると考え、形態素間の関係を考慮しなかった。
しかし、文を係り受け解析し、形態素の感情属性に重み付けをすることで、より高い精度が得られると考えた。



\section{まとめ}\label{sec:summary}
  %考察と原因の解明
本研究は、顔文字辞書と顔文字辞書を構築し、入力文の感情に見合った顔文字を推薦するシステムを提案すること目的とした。
\sref{sec:def}では、本研究で用いたナイーブベイズ分類について述べた。ナイーブベイズ分類は、過去の事例をもとに未知の文書があらかじめ与えられている場合にどのカテゴリに帰属するかを決定する分類手法である。
\sref{sec:theorem}では、ナイーブベイズ分類を用いて、Twitter から収集した文をもとにした感情語辞書と顔文字辞書の構築手法を述べた。このとき、学習データとして先行研究の感情コーパスを使用した。文中の形態素 $w_i$ が出現した際に、各感情 $e_2,e_2, \cdots ,e_7$ それぞれに帰属する確率を求めた。そして、これらの辞書を用いて入力文の感情を推定し、感情に適切な顔文字を推薦するシステムを提案した。このとき、出力された感情に最も適切な顔文字を選出するため、入力文の感情語と顔文字辞書に登録された感情の感情ベクトルのハミング距離を求め、ハミング距離が最も小さいものを入力文に付与した。
\sref{sec:experiment}では、顔文字推薦システムの精度を検証するため、5名の被験者に感情生起文を各感情につき5文ずつに入力してもらった。入力文に対して付与された顔文字が適切であれば正解、適切でなければ不正解とし、正解率を出した。この正解率を顔文字推薦システムの精度とした。この実験の結果、顔文字推薦システムの精度は 84.0\% であった。
\sref{sec:analyze}では、実験結果をもとに、顔文字推薦システムの問題点とその原因について述べた。適切な感情が抽出されていても顔文字が不適切であるという問題と、未知の若者言葉が感情語として文中に出現した場合に感情を推定できないという問題、感情語辞書に大きな偏りがある問題、そして形態素の前後の関係を考慮してないない問題を明らかにした。今後はこれらの問題点を考慮し、さらなる精度向上を目指す必要がある。

 %参考文献ファイル設定
 \clearpage
 \bibliographystyle{ib_plain}
 \bibliography{lastthesisref} %bibファイルネーム
\end{document}
